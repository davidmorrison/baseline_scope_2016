\section*{Shortened HCal}
\label{ohcal_short}

\textbf{Cost delta: --\$0.7M}
 
We investigated shortening the outer HCal (nominal length 631~cm) to
541~cm.  For an HCal of the nominal length, the outer corner of the
HCal lies along $\eta = 1.0$ while then inner corner of the HCal is at
$\eta = 1.3$.  For a reduced length calorimeter, the outer corner of
the HCal lies along $\eta = 0.89$ while then inner corner of the HCal
is at $\eta = 1.19$.  This is a minimum length for the outer HCal, as
supporting the weight of the calorimeter requires it to extend beyond
the end of the solenoid cryostat.

The effect on sphenix physics stems directly from the reduced
acceptance.  The smaller acceptance catches fewer jets and high $p_T$
hadrons and dijets, and it also reduces the acceptance for systematic
studies of the response.  The former issues can be determined in a
straightforward manner; the effect of the latter issue is guided by
relevant experience.

From the perspective of the project, save on not only mechanics but
also scintillator tiles and electronics channels.  The shorter OHCal
will be easier to make, handle, transport and assemble.  It will be
more rigid.  The detector cross section will be unchanged so
engineering revisions will be minimal (the project estimates that it
would take four weeks). The R\&D done to this point will be valid.  We
can advance to the next step in prototype work.  Restoring the
acceptance with endcap calorimeters is conceivable, but would require
immediate consideration in the engineering of the main part of the
detector.  In addition, the procurement process for the endcap steel
would need to begin very soon, as it would have the similar long
lead-time issues as the steel for the current design.

\subsection*{Considerations}

Shortening the outer HCal steel also implies moving the flux doors at
each end of the experiment inward by the same 45~cm.  This is needed
so that the strong magnetic field is well-contained and kept away from
the electonics racks on the carriage, allowing personnel to get to
those racks for routine or exceptional maintenance during the run
without requiring the magnet to be deenergized.  Moving the flux doors
inward by this amount significantly reduces the space potentially
available for instrumentation beween the end of the TPC and the flux
door, as would be required for forward spin and cold QCD measurements.
This would be seen particularly negatively by the members of the
collaboration interesting in spin physics and by those intereseted in
the use of sPHENIX as a day-1 EIC detector.
