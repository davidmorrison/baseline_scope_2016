\chapter*{sPHENIX configurations}
\label{configurations}
\setcounter{page}{1}

\section{BWC I configuration}
\label{sec:bwc-i-configuration}

The collaboration has developed a consensus for a preferred detector
configuration.  It is the result of prioritizing high performance
tracking, even though it requires difficult non-recoverable
compromises to the calorimeter stack. 

Relative to the nominal baseline configuration, configuration ``A''
makes the following changes:

\begin{description}
\item[Thinned oHCal] The outer HCal is thinned by 20~cm, or about one
  interaction length.  Savings: \$2M.
\item[Reduced EMCal segmentation] The EMCal segmentation is coarsened
  to $\Delta\phi \times \Delta\eta =  0.03 \times 0.03$.  Savings:
  \$2M.
\item[Reduced DAQ/Trigger scope] By aggressively reusing existing
  trigger counters and DAQ hardware, the scope of Trigger/DAQ is
  focused narrowly on new data collection modules. Savings: \$1.5M.
\item[Sparsified TPC readout] The TPC electronics reads out alternate
  pad rows. Savings: \$0.5M.
\item[Two layer MAPS inner tracker] Not reusing the VTX pixels is a
  minor savings; the dominant effect of this change is the added cost
  of two layers of highly performant MAPS pixels closely based on the
  ALICE inner barrel design. Additional cost: \$3M.
\end{description}

The net effect of this is a savings of \$3M relative to the cost of
the nominal baseline configuration.

