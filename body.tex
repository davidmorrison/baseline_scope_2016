\chapter{Overview}
\label{configurations}
\setcounter{page}{1}

\section{Introduction}
In this document the sPHENIX collaboration answers a charge~(see
Appendix~\ref{charge}) from BNL ALD Berndt Mueller to develop a
baseline design scope that provides a compelling physics program
within the constraints of possible DOE funding redirected from RHIC
operations. The document describes a reference design aimed at a
compelling program focused on three science drivers: jet structure,
heavy-flavor jet production and $\Upsilon$ spectroscopy. We then
provide a comprehensive list of de-scoping options relative to the
reference design for each of the main subdetector systems, and
describe the associated cost savings, the engineering and schedule
impact and impact on key performance measures related to the three
science drivers. Based on these criteria, we develop examples of
rank-ordered lists of de-scoping options including the cumulative
cost-savings and science impact.

The Laboratory has indicated that a maximum of \$75M in redirected DOE
RHIC operations funds could be used to construct sPHENIX.  Based on
guidance from the project, and confirmed by discussion with the ALD,
the charge is equivalent to seeking a reduction in discretionary M\&S
costs by nearly \$4M in FY16 dollars compared to the detector design
described in the 2015 Director's Cost and Schedule review.  In this
document we will therefore exclusively focus on these M\&S savings,
excluding non-discretionary infrastructure items such as cryogenics,
magnet, central pedestal and flux doors.

We first describe the key science drivers behind sPHENIX, the
methodology we have used to evaluate the physics performance and the
cost savings associated with various detector configurations.  We then
describe a reference configuration that would enable the collaboration
to address the full program of physics detailed in the sPHENIX
proposal which was reviewed by the DOE convened panel in April 2015.
We provide an overview of the options we have considered for
reconfiguration of the sPHENIX detector, and discuss their physics
capabilities relative to the reference configuration.  Lastly, we
provide a list of changes, prioritized according to considered input
from the collaboration, and we detail two scenarios that very
significantly cut the M\&S costs while retaining a program of
compelling physics and enabling a path forward to recover a full
detector should additional funds be identified.  It should be noted
that all reconfigurations, alterations, and deletions we have
identified severely and negatively impact the science program of
sPHENIX.  The sPHENIX scientific collaboration working closely with
the project, through extensive discussions --- including a three-day
in-person collaboration meeting at BNL held May 18--20 --- has
wrestled with these issues to reach the consensus respresented here.
