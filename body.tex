\chapter*{Overview}
\label{configurations}
\setcounter{page}{1}

\section{Introduction}
In this document the sPHENIX collaboration answers a charge~(see
Appendix~\ref{charge}) from BNL ALD Berndt Mueller to develop
a baseline design scope that provides a compelling phsyics program
within the constraints of possible DOE funding redirected from 
RHIC operations. The document describes a reference design aimed
at a compelling program focussed on three science drivers: jet structure,
heavy-flavor jet production and $\Upsilon$ spectroscopy. We then
provide a comprehensive list of re-scoping options relative 
to the reference design for each
of the main subdetector systems, and describe the associated
cost savings, the engineering and schedule impact and impact on 
key performane measures related to the three science 
drivers. Based on these criteria, we develop examples of 
rank-ordered lists of re-scoping options including the 
cumulative cost-savings and science impact.

The total amount of redirected DOE funds available is quoted as \$75M in FY16 dollars.
Based on discussions between ALD, project and collaboration, the charge can
however more narrowly interpreted as aiming at a reduction in discretionary M\&S costs 
by \$4M in FY16 dollars compared to the detector design described
in the 2015 directors cost and schedule review. In this document we will
therefore exclusively focus on these M\&S savings, excluding non-discretionary 
infrastructre items such as cryogenics, magnet, central pedestal and flux doors.

