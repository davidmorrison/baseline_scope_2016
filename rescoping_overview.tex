
\section{Overview of re-scoping options}
\label{sec:overview-re-scoping}

In this section we provide an overview of the rescoping options that
have been considered, even ones that are extremely unattractive due to
the enormous loss of science capbilities, the introduction of project
schedule risk, or other negative impacts on collaboration interests.
For each item, we state briefly the cost savings expected (in direct
FY16\$) and the major science and project impacts.  The detailed
evaluation of each of these options is shown in the Appendices.

While we have studied significant changes to each of the major subsystems,
including tracker, EMCal and HCal, complete removal of one of 
these elements is not compatible with the overarching science goal
of developing a state-of-the-art jet detector at RHIC. The current 
literature on LHC jet measurements demonstrates the essential 
contributions of each of these subsystems to state-of-the-art jet 
measurements. Achieving similar performance at RHIC is 
even more demanding than at LHC and will rely on the combination 
of all detector systems.

We considered reducing the depth of the outer HCal by 20~cm
(approximately one interaction length).  The cost savings would be
\${\thinnedOuterHCal}M.  This would not be a recoverable change. The
impact on the jet energy response would be moderate.  This change
would result in a significant increase in risk to the project
schedule.

Shortening the outer and inner HCal, reducing their $\eta$ coverage to
$\pm0.9$ would produce savings of \${\shortenedHCal}M.  This would not be a
recoverable change.  It would reduce the acceptance for jets, directly
reducing statistics, and generating increased difficulties for
studying and controlling systematic uncertainties.  It would require
re-engineering of the support structure for the experiment and would
necessitate moving each of the flux return doors in toward the IP by
45~cm.  This would reduce the available volume between the barrel tracker 
and the flux return doors significantly. In such a configuration it 
would not be possible to eventually realize the forward hadron arm 
required for an EIC detector as there would be insufficient space for 
both tracking and particle identification.

We considered leaving out the inner HCal entirely.  The inner HCal has
three important functions in sPHENIX.  It provides an interaction
length of active detector near the shower max of a hadron shower,
serves as a tail-catcher for EM shower energy leaking out of the back
of the EMCal, and is the structural support for the EMCal.  The
structural support function could be replaced with approximately three
months of engineering work and \$100k in new M\&S cost.  The net
savings would be \${\noInnerHCal}M.  Without the inner HCal, the
EM-shower tail-catching function would have to be provided by the
outer Hcal.  Due to the distance and the amount of inactive material
between the EMCal and outer HCal, the outer HCal performs poorly as a
tail-catcher.  Finally with no inner HCal, a hadron that starts
showering in the EMCal would emerge into an air gap containing a 1.5~T
field followed by the magnet plus cryostat and then the outer HCal.
The shower would be distorted in a way that must be better understood
before we can quantify its effect on energy resolution, jet
identification or the ability to subtract the underlying event.

We have considered reducing the granularity of the EMCal, reducing the
number of towers and their associated electronics.  This produces a
savings of \$1.8M.  The would not be a recoverable change.  The major
effect on the science would be a worsened $e/\pi$ separation,
particularly in Au+Au collisions, effectively reducing the $\Upsilon$
statistics.  It would require both the EMCal and the inner HCal
structures to be re-engineered.  Practicalities of producing larger
towers would have to investigated and it is possible that further test
beam time would be needed to validate the design.

We have considered ganging together the electronic readouts of 2x2
collections of EMCal towers. This produces savings very similar to
that of reducing the granularity of the detector, \$1.8M.  It would be
a recoverable change.  The major effects on the science are similar to
the reduced granularity option, except that the effective tower size
is even larger.  It introduces the risk that funding to recover the
deferred electronics will not be identified and the experiment is left
with the even coarser segmentation.  The project impact is moderate
and would require engineering the electronic ganging and additional QA
checks during construction on the relative gain of ganged SiPMs.

For each of the EMCal segmentation reducing options we further
considered reducing the pseudorapidity coverage to $|\eta| < 0.6$.
The savings in both cases is very similar at \${\reducedEMCaleta}M.
This is a recoverable change.  The science effects are reduced
acceptance for photons and for the $\Upsilon$ family and the
introduction into the acceptance of varying material thickness, which
introduces additional complications in controlling the jet response
systematics. The effect on the project is moderate.

We considered sparsifying the TPC readout, instrumenting only every
other row of readout pads.  This is expected to save
\${\reducedTPCreadout}M.  It is a potentially recoverable change. The
effect on key sPHENIX science priorities is moderate, however, the
capability for $dE/dx$ measurements is worsened.  The effect on the
project is minimal.

We considered reducing the elements of the DAQ/Trigger that would be
built or refreshed.  Reusing an existing beam-beam counter could save
\${\reuseBBC}M and reducing the refresh of DAQ hardware could save
\${\reducedDAQ}M.  These are recoverable changes.  The effect on the
science is tolerable, though it introduces the risk of delays in
analyzing acquired data, and could reduce the volume of recorded
minimum bias data significantly. The effect on the project is minimal.

We have considered leaving the reconfigured VTX out of sPHENIX
entirely.  This is expected to save \${\noVTX}M.  The continued experience
with the VTX pixels in Run-16 --- in terms of functioning area and
operational stability --- has added to the collaboration's
determination that the reconfigured pixels cannot deliver sPHENIX
physics. 

We have considered introducing a two-layer MAPS-based inner tracker
and removing any dependence on the reuse of the PHENIX VTX pixels.
This would cost \${\twoLayerMAPS}M.  The effect on the science case is very
positive, preserving HF--tagged jet and open HF capabilities. 

We have also considered the physics and cost effect of one-layer of
MAPS-based inner pixels.  Implementing a single pixel layer would cost
\${\oneLayerMAPS}M.  As with the two-layer MAPS option, it removes any
dependence on the reuse of the PHENIX VTX pixels.  In conjunction with
knowledge of the primary vertex obtained using beam-beam trigger
counters, a single layer of pixels with a short integration time (the
ALPIDE chip aims for 4~$\mu s$ compared to 185.6~$\mu s$ for the STAR
HFT) might be able to provide the tracking support needed by the TPC,
though this would take further study to establish conclusively.
Compared to the two-layer inner tracker, however, one loses the
HF--tagged jet and open HF capabilities.






