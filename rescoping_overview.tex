
\section{Overview of re-scoping options}
\label{sec:overview-re-scoping}

In this section we provide an overview of the rescoping options that
have been considered.  For each item, we state briefly the cost
savings expected (in direct FY16\$) and the major science and project
impacts.  The detailed evaluation of each of these options is shown in
the Appendix.

We considered reducing the depth of the outer HCal by 20~cm
(approximately one interaction length).  The cost savings would be
\$0.4M.  This would not be a recoverable change. The impact on the jet
energy response would be moderate.  This change would result in a
significant increase in risk to the project schedule.

Shortening the outer and inner HCal, reducing their $\eta$ coverage to
$\pm0.9$ would produce savings of \$0.7M.  This would not be a
recoverable change.  It would reduce the acceptance for jets, directly
reducing statistics, and generating increased difficulties for
studying and controlling systematic uncertainties.  It would require
re-engineering of the support structure for the experiment and would
necessitate moving each of the flux return doors in toward the IP by
45~cm.  This would eliminate most of the clear volume between the
barrel tracker and the doors and would likely make a forward tracker
impossible.

We considered reducing the granularity of the EMCal, reducing the
number of towers and their associated electronics.  This produces a
savings of \$1.8M.  The would not be a recoverable change.  The major
effect on the science would be a worsened $e/\pi$ separation,
particularly in Au+Au collisions, effectively reducing the $\Upsilon$
statistics.  It would require both the EMCal and the inner HCal
structures to be re-engineered.  Practicalities of producing larger towers
would have to investigated and it is possible that further test beam
time would be needed to validate the design.

We considered ganging together the electronic readouts of 2x2
collections of EMCal towers. This would produce savings of \$1.7M.  It
would be a recoverable change.  The major effects on the science are
similar to the reduced granularity option, except that the effective
tower size is even larger.  There is a risk that funding to recover
the deferred electronics will not be identified and the experiment is
left with the even coarser segmentation.  The project impact is
moderate and would require engineering the electronic ganging and
additional QA checks during construction on the relative gain of
ganged SiPMs. 

For each of these EMCal options we further considered reducing the
pseudorapidity coverage to $|\eta| < 0.7$.  The savings in both cases
is very similar at \$2M.  This is a recoverable change.  The science
effects are reduced acceptance for photons and for the $\Upsilon$
family and the introduction into the acceptance of varying material
thickness, which introduces additional complications in controlling
the jet response systematics. The effect on the project is moderate.

We considered sparsifying the TPC readout, instrumenting only every
other row of readout pads.  This is expected to save \$0.5M.  It is a
potentially recoverable change. The effect on key sPHENIX science
priorities is moderate, however, the capability for $dE/dx$
measurements is worsened.  The effect on the project is minimal. 

We considered reducing the elements of the DAQ/Trigger that would be
built or refreshed.  This is expected to save \$1.5M.  It is a
recoverable change.  The effect on the science is minimal, though it
introduces the risk of delays in analyzing acquired data. The effect
on the project is minimal. 

Finally, we have considered introducing a MAPS-based inner tracker and
removing any dependence on the reuse of the PHENIX VTX pixels.  This
produces an anti-savings of \$3M.  The effect on the science case is
very positive, preserving HF--tagged jet and open HF capabilities. 





