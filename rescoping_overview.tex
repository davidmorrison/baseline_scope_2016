
\section{Overview of re-scoping options}
\label{sec:overview-re-scoping}

In this section we provide an overview of the rescoping options that
have been considered.  For each item, we state briefly the cost
savings expected (in direct FY16\$) and the major science and project
imapcts.  The detailed evaluation of each of these options is shown in
the Appendix.

We considered reducing the depth of the outer HCal by 20~cm
(approximately one interaction length).  The cost savings would be
\$0.4M.  The impact on the jet energy response would be moderate.
This change would result in a significant increase in risk to the
proect schedule. 

Shortening the outer and inner HCal, reducing their $\eta$ coverage to
$\pm0.9$ would produce savings of \$0.7M.  This would reduce the
acceptance for jets, directly reducing statistics, and generating
increased difficulties for studying and controlling systematic
uncertainties.  It would require reengineering of the support
structure for the experiment and would necessitate moving each of the flux
return doors in toward the IP by 45~cm.  This would eliminate most of
the clear volume between the barrel tracker and the doors and would
likely make a forward tracker impossible.


