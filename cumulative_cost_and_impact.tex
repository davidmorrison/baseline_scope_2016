
\section{Cumulative cost and impact}
\label{sec:cumul-cost-impact}

Table~\ref{tab:rescoping_options} and \ref{tab:calo_options} list re-scoping options in the order which we
believe preserves with the highest priority the most compelling
elements of the sPHENIX science program.  With each item is a cost
change if that option is pursued, along with a cumulative total of
those cost changes. Table~\ref{tab:calo_options} lists HCAL-related options that were considered. 
However, these options are disfavored due to the significant 
impact they have on the project schedule, infrastructure and damage to the 
physics program, especially for jets above $|\eta| > 0.7$ when combined with a reduction in the EMCal rapidity coverage.

\renewcommand{\arraystretch}{1.4}
\begin{table}
  \caption{Ordered list of re-scoping options for the sPHENIX
    detector. The column labeled ``$\Delta$'' shows the cost delta
    associated with the particular option. The column labeled
    ``$\Sigma$'' is the running sum of the cumulative cost changes.}
  \label{tab:rescoping_options}
  \centering
  \begin{tabular}{ld{3.1}d{3.1}l}
    \toprule
    Option & \multicolumn{1}{c}{$\Delta$ (\$M)} &
    \multicolumn{1}{c}{$\Sigma$ (\$M)} & Impacts \\
    \midrule
    \multirow{3}{*}{two-layer MAPS inner barrel} &  \multirow{3}{*}{\twoLayerMAPS} &
    \multirow{3}{*}{3.0} & preserve HF jet tagging \\
    & & & support for TPC tracks \\
    & & & greater $z_\mathrm{vertex}$ acceptance \\
    \midrule
    \multirow{1}{*}{No VTX reuse} & \multirow{1}{*}{-\noVTX} &
    \multirow{1}{*}{2.8} & see MAPS above \\
    \midrule
    \multirow{2}{*}{Sparsified TPC readout} & \multirow{2}{*}{-\reducedTPCreadout} &
    \multirow{2}{*}{2.3} & worsened $\Upsilon$ mass resolution \\
    & & & worsened $dE/dx$ resolution \\
    \midrule
    \multirow{1}{*}{Re-use beam-beam counter} & \multirow{1}{*}{-\reuseBBC} & \multirow{1}{*}{1.8} & potentially worsened vertex
    resolution \\ 
    \midrule
    \multirow{3}{*}{Reduced DAQ refresh} & \multirow{3}{*}{-\reducedDAQ} &
    \multirow{3}{*}{1.3} & possible reduction of min. bias data sample
    \\ 
    & & & potential delay to analysis \\
    & & &  operational risk of past-EOL equipment \\
    \midrule
    2x2 EMCal tower ganging & \multirow{3}{*}{-\reducedEMCalsegmentation} & \multirow{3}{*}{-0.5} & worsened $e/\pi$ separation \\
    or & & & reduced $\Upsilon$ statistics \\
    coarsened segmentation & & & worsened $\gamma$ isolation
    performance \\
    \midrule
    \multirow{3}{*}{EMCal reduced $\eta$} & \multirow{3}{*}{-\reducedEMCaleta} & \multirow{3}{*}{-2.5} & reduced $\Upsilon$
    statistics \\
    & & &  reduced direct photon statistics \\
    & & &  worse control of jet unfolding systematics\\
    \bottomrule
  \end{tabular}
\end{table}

\begin{table}[hbt]
  \caption{List of considered HCal modifications.  Considered in
    isolation, the effect of any one of these alterations on the
    physics capabilities is moderate.  However, a combination of these
    options, or any of them in combination with alterations to the
    EMCal has very negative effects on the physics capabilities of the
    experiment.  These alterations also introduce significant new
    schedule risk and have negative implications for potential forward
    instrumentation upgrades or EIC suitability.}
  \label{tab:calo_options}
  \centering
  \begin{tabular}{ld{3.1}l}
    \toprule
    Option & \multicolumn{1}{c}{$\Delta$ (\$M)} & Impacts \\
    \midrule
    \multirow{4}{*}{Thinned outer HCal} & \multirow{4}{*}{-0.4} & high $p_T$ punch-through \\
    & & worsened jet energy response \\
    & & extensive reengineering required \\
    & & risk to schedule \\
    \midrule
    \multirow{2}{*}{HCal reduced $\eta$} & \multirow{2}{*}{-0.7} & reduced acceptance for jets \\
    & &  no clear volume for possible forward tracking \\
    \midrule
    \multirow{3}{*}{Remove inner HCal} & \multirow{3}{*}{-1.0} & worsened jet response \\ 
    & &  jet flavor bias \\
    & &  worsened $e/\pi$ in p+p \\
    \bottomrule
  \end{tabular}
\end{table}
