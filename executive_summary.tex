\section*{Executive Summary}
\label{executive_summary}
\setcounter{page}{1}

\nocite{*}

In this document the sPHENIX collaboration answers a charge~(see
Appendix~\ref{charge}) from BNL ALD Berndt Mueller to detail a
configuration for the sPHENIX detector that conforms to specific
funding guidance and to describe the program of compelling physics
accessible to that particular configuration.

sPHENIX is a next-generation RHIC experiment providing world-class
measurement capabilities of key scale-sensitive observables to
understand the dynamics of the strongly coupled quark gluon plasma.
It uniquely provides measurements complementary to those being
obtained at the LHC by working at a collision energy near $T_c$, where
the medium coupling is believed to be strongest and the corresponding
dynamical effects the most pronounced.

The physics aims of sPHENIX have been endorsed broadly and repeatedly.
The top recommendation of the September 2014 “Phases of QCD Matter”
Town Meeting at Temple University reads in part, ``implementation of
new capabilities of the RHIC facility (a state-of-the-art jet detector
such as sPHENIX and luminosity upgrades for running at low energies)
[is] needed to complete its scientific mission,''.  The top
recommendation of ``The Hot QCD White Paper: Exploring the Phases of
QCD at RHIC and the LHC'', includes the statement, ``implementation of
new capabilities of the RHIC facility needed to complete its
scientific mission: a state-of-the-art jet detector such as sPHENIX
and luminosity upgrades for running at low energies.'' Both of these
documents provided carefully considered community input for the
development of the most recent NSAC Long Range Plan, ``The 2015 Long
Range Plan for Nuclear Science,'' which was officially accepted by the
DOE Office of Nuclear Phyics in October 2015, and reads in part:

\blockquote{There are two central goals of measurements planned at
  RHIC, as it completes its scientific mission, and at the LHC: (1)
  Probe the inner workings of QGP by resolving its properties at
  shorter and shorter length scales. The complementarity of the two
  facilities is essential to this goal, as is a state-of-the-art jet
  detector at RHIC, called sPHENIX. (2) Map the phase diagram of QCD
  with experiments planned at RHIC.}

In addition to these community-wide endorsements of the aims of
sPHENIX, the detailed science case presented in the sPHENIX proposal
had a successful science review in April 2015 conducted by the DOE
Office of Nuclear Physics.




