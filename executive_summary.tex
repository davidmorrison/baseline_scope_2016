\section*{Executive Summary}
\label{executive_summary}
\setcounter{page}{1}

\nocite{*}

In this document the sPHENIX Collaboration answers a charge~(see
Appendix~\ref{charge}) from BNL ALD Berndt Mueller to develop a
baseline design scope that provides a compelling physics program
within the constraints of possible DOE funding redirected from RHIC
operations. This document describes a reference design based on the
calorimeter configuration in the sPHENIX pCDR and a tracking system
combining a MAPS (monolithic active pixel sensors)-based inner tracker
and a TPC (time projection chamber) outer tracker. We demonstrate the
performance of the reference configuration for a program focused on
three science drivers: jet structure, heavy-flavor jet production and
$\Upsilon$ spectroscopy.  We then provide a comprehensive list of
de-scoping options for the calorimeters and the inner and outer
trackers. For each change we describe the associated cost savings, the
engineering and schedule impact based on estimates by the sPHENIX
Project Management team in consultation with the engineering team and
subsystem experts at the level of detail achievable in the time
allowed.  The physics impact of each de-scoping option is studied for
a focussed set of performance criteria related to the three science
drivers, using simulations ranging from generator level studies to
full detector simulations. Based on the current status of these
studies, we develop a rank-ordered list of de-scoping options, to
allow balancing of cost savings and science performance. Two possible
configurations that aim to optimize this balance are shown as examples
in Table~\ref{tab:scenarios}, with cost savings shown relative to the
sPHENIX pCDR configuration.

\begin{table}[hbt]
  \centering
  \caption{Cost reduction scenarios identified by the scientific
    collaboration in consultation with the project that
    signficantly reduce the M\&S costs while preserving a 
    compelling science program.  Both scenarios involve very serious
    cuts to detectors and represent very unfortunate degradations in
    capability. Both scenarios have significantly worsened $e/\pi$
    separation, acceptance for $\Upsilon$s and photons, and suffer a
    longer time before reconstructed data would be available.  Both
    scenarios identify a path to restore capabilities should
    additional funding become available.  The scenario on the left retains
    the ability to identify displaced tracks, preserving HF-tagged
    jet capability. The scenario on the right sacrifices even this key
    physics capability.    
    Cost differences are in FY16 \$M, relative to the sPHENIX pCDR
    configuration.}   
  \begin{tabular}{ld{3.1}|ld{3.1}}
    \toprule
    \multicolumn{1}{c}{Scenario A} & \multicolumn{1}{c}{$\Delta$} &
    \multicolumn{1}{c}{Scenario B} & \multicolumn{1}{c}{$\Delta$}  \\
    \midrule
    two-layer MAPS inner barrel & +\twoLayerMAPS & 
    one-layer MAPS inner barrel & +\oneLayerMAPS \\
    no reuse of VTX & -\noVTX & no reuse of VTX & -\noVTX \\
    reduce TPC readout & -\reducedTPCreadout & 
    reduce TPC readout & -\reducedTPCreadout \\
    reduce EMCal segmentation & -\reducedEMCalsegmentation &  
    reduce EMCal segmentation & -\reducedEMCalsegmentation \\
    reduce EMCal $\eta$ acceptance & -\reducedEMCaleta & 
    further reduce EMCal $\eta$ acceptance &  -\reducedEMCaletaMore \\
    reduce DAQ refresh & -\reducedDAQ & reduce DAQ refresh & -\reducedDAQ \\
    reuse beam-beam trigger counter & -\reuseBBC & 
    reuse beam-beam trigger counter & -\reuseBBC \\
    \midrule
    \multicolumn{1}{r}{Total} & -2.5 & \multicolumn{1}{r}{Total} & -3.6 \\
    \bottomrule
  \end{tabular}
  \label{tab:scenarios}
\end{table}
