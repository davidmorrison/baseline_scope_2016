\section*{Executive Summary}
\label{executive_summary}
\setcounter{page}{1}

\nocite{*}

In this document the sPHENIX collaboration answers a charge~(see
Appendix~\ref{charge}) from BNL ALD Berndt Mueller to develop
a baseline design scope that provides a compelling physics program
within the constraints of possible DOE funding redirected from 
RHIC operations. The document describes a reference design based
on the calorimeter configuration in the sPHENIX pCDR and a tracking system
combining a 3-layer MAPS inner tracker and a TPC outer tracker. We
demonstrate the performance of the reference configuration for a
a compelling physics program focused on three science drivers: jet structure,
heavy-flavor jet production and $\Upsilon$ spectroscopy. 
We then provide a comprehensive list of re-scoping options for each
of the main subdetector systems. For each change we describe the associated
cost savings, the engineering and schedule impact and impact on 
key performance measures related to the three science 
drivers. Based on these criteria, we develop a 
rank-ordered list of re-scoping options, including the 
cumulative cost-savings and science impact, to allow determination
of an optimal balance of cost savings and science performance.


