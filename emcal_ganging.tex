\section{EMCal}

The EMCal accounts for a significant portion of the M\&S budget.  It
also enables the key $\Upsilon$ and direct $\gamma$ and $\gamma+$jet
capabilities of sPHENIX.

\subsection{Reduction in EMCal Segmentation}
\label{emcal_ganging}
\label{emcal_segmentation}

\textbf{Cost delta: -\$1.7M}

We have considered two options to reduce the electronics cost
associated with the EMCal by reducing the segmentation.  One is
ganging together groups of 2x2 towers in the reference configuration
tower size ($\Delta\eta\times\Delta\phi = 0.024\times0.024$, leading
to an effective segmentation of $0.048 \times 0.048$) and the other is
to make a permanent reduction in the EMCal granularity by making the
towers bigger (one such configuration could be $0.033 \times 0.028$ in
$\Delta \phi x \Delta \eta$), reducing the number of towers by about
38\%. Both options for reducing the segmentation are estimated to
yield nearly the same cost savings.

Ganging together towers means that the R\&D done to date --- both on
production aspects as well as performance in the test beam --- remains
valid.  It does introduce a ``mortgage'' in which the collaboration
would try to identify funds to buy the additional electronics needed
to allow towers be read out individually.

Reducing the EMCal granularity via larger towers would reduce the cost
of the readout electronics, but would likely require increasing the
number of SiPMs per tower to preserve the light collection
uniformity. An advantage of the second option is that this would
provide a finer segmentation than the ganging option, however it would
not allow the reference configuration segmentation to be recovered.
Because this reduced granularity would be a permanent change, it does
not result in a mortgage.

The EMCal R\&D to this point has focused on the production of towers
of the $0.024\times 0.024$ size in the reference configuration.
Increasing the tower size introduces potential production issues in
the tower construction and performance. Additional R\&D will need to
be done to determine if larger towers can be made as efficiently and
uniformly as in the reference configuration and that the light can be
collected from larger area lowers with sufficient uniformity.  If this
option is pursued we would need to pursue R\&D on the larger size
towers to address the concerns described above and based on those
finding to decide which option provides the most feasible construction
option.  This R\&D would delay the overall EMCal schedule by
approximately six months and might require additional prototyping
beyond that which is currently in the schedule.
