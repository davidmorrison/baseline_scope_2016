\section{Summary}

In this document the sPHENIX collaboration has answered a charge~(see
Appendix~\ref{charge}) from BNL ALD Berndt Mueller to develop a
baseline design scope that provides a compelling physics program
within the constraints of possible DOE funding redirected from RHIC
operations. The document describes a reference design based on the
calorimeter configuration in the sPHENIX pCDR and a tracking system
combining a MAPS inner tracker and a TPC outer tracker. We demonstrate
the performance of the reference configuration for a program focused
on three science drivers: jet structure, heavy-flavor jet production
and $\Upsilon$ spectroscopy.  We have provided a comprehensive list of
de-scoping options for each of the main subdetector systems. For each
change we have described the associated cost savings, the engineering
and schedule impact and impact on key performance measures related to
the three science drivers. Based on these criteria, we have developed
a rank-ordered list of de-scoping options, including the cumulative
cost-savings and science impact, to allow determination of an optimal
balance of cost savings and science performance. An important
conclusion from these studies is that the sPHENIX design has been
well-optimized and is tightly integrated. It is not possible to make
significant reductions in the detector configuration without
associated major impacts on the key physics driving the sPHENIX
program.
