\section{Methodology and document structure}

The discretionary M\&S expenditures are linked to six major subsystems: outer HCal, inner HCal, EMCal, outer tracker, inner tracker and DAQ/trigger, 
following the structure established by the sPHENIX project. For each of the five subsystems, we solicited input from the collaboration, the 
project leadership, the three topical groups and subsystem experts to assemble a comprehensive list of re-scoping options aimed at reducing 
individual subsystem costs. For each identified option, cost savings and the engineering, design and schedule impact were estimated by project 
leadership in consultation with the engineering team and subsystem experts. The resulting cost, engineering and design impact, at the 
level of detail achievable in the time allowed, is described in Appendix A.

The large number of identified options precludes establishing simulation geometries for all possible combinations of subsystem configurations
and running full GEANT simulations for these combined configurations. Rather, we studied the impact on key physics performance figures
related to three science drivers for each option individually and for select combinations of options when they were individually found to
have non-negligible impact on the same performance characteristics. For the various options, the performance studies were done at different 
levels of detail, ranging from generator level studies of loss of statistical precision, to single particle \geant + reconstruction studies 
of resolution and bias effects to full central Au+Au HIJING \geant + reconstruction studies. Where applicable, the physics impact of various
changes was also evaluated based on experience with corresponding
measurements at RHIC and the LHC. In particular for the MAPS inner tracking 
option and the TPC outer tracker, we consulted with the project managers in the ALICE upgrade program, Luciano Musa (CERN, ALICE ITS upgrade 
manager) and Harald Appelshauser (Frankfurt, TPC project manager) and with experts from the STAR collaboration (Flemming Videbaek, HFT and
iTPC project manager and Gene van Buren, STAR TPC expert). The resulting science impact evaluation for all proposed subsystem changes 
is described in Appendix B.

The following sections of the document are organized as follows: We briefly describe the reference design used to evaluate the relative
performance impact of each subsystem option. Next we provide a concise summary of the cost, engineering, design and schedule impact and 
science impact of each option. This is followed by a rank-ordered table of options, from least to most undesirable, based on our 
evaluation of savings vs.\ engineering and science impact. Following a summary, we provide an extended discussion of each option's impact
in the Appendices A (cost, engineering, schedule) and B (physics performance).


