\section{sPHENIX science drivers}

sPHENIX is a next-generation RHIC experiment providing world-class
capabilities for multi-scale studies of the microscopic nature
of the strongly coupled quark gluon plasma.
It uniquely provides measurements complementary to those being
obtained at the LHC by working at a collision energy near $T_c$, where
the medium coupling is believed to be strongest and the corresponding
dynamical effects the most pronounced.

The physics aims of sPHENIX have been endorsed broadly and repeatedly.
The top recommendation of the September 2014 “Phases of QCD Matter”
Town Meeting at Temple University reads in part, ``implementation of
new capabilities of the RHIC facility (a state-of-the-art jet detector
such as sPHENIX and luminosity upgrades for running at low energies)
[is] needed to complete its scientific mission,''.  The top
recommendation of ``The Hot QCD White Paper: Exploring the Phases of
QCD at RHIC and the LHC'', includes the statement, ``implementation of
new capabilities of the RHIC facility needed to complete its
scientific mission: a state-of-the-art jet detector such as sPHENIX
and luminosity upgrades for running at low energies.'' Both of these
documents provided carefully considered community input for the
development of the most recent NSAC Long Range Plan, ``The 2015 Long
Range Plan for Nuclear Science,'' which was officially accepted by the
DOE Office of Nuclear Physics in October 2015, and reads in part:

\blockquote{There are two central goals of measurements planned at
  RHIC, as it completes its scientific mission, and at the LHC: (1)
  Probe the inner workings of QGP by resolving its properties at
  shorter and shorter length scales. The complementarity of the two
  facilities is essential to this goal, as is a state-of-the-art jet
  detector at RHIC, called sPHENIX. (2) Map the phase diagram of QCD
  with experiments planned at RHIC.}

In addition to these community-wide endorsements of the aims of
sPHENIX, the detailed science case presented in the sPHENIX proposal
had a successful science review in April 2015 conducted by the DOE
Office of Nuclear Physics.


For the physics studies presented in this document we considered three
main science drivers. These were selected to represent the main known 
experimental approaches to the science case discussed in the NP LRP and
to provide a comprehensive test of all aspects of the expected detector
performance. They are:
\begin{enumerate}
\item {\bf Jet structure:} Modifications of jet structure provide a unique test
of the microscopic nature of the sQGP, as each jet itself represents a multi-scale
object in angular and momentum space. Corresponding measurements depend
sensitively on the performance on the acceptance and performance 
of the calorimeter system for jet finding and jet energy determination, the
EMCal for measuring isolated photons as jet energy tags and on the tracking
system to provide high resolution charged particle information to characterize
the jet structure jet-by-jet.
\item {\bf Heavy-flavor tagged jets:} Measurements of heavy-flavor hadrons 
and jets over a wide kinematic range are critical for studies of
the flavor dependence of parton interactions with the medium for and for 
disentangling the effects of radiative vs.\ collisional energy loss processes.
In addition to the general jet finding performance, these measurements depend 
critically on the performance of the inner tracking system for displaced 
tracks and vertices needed for high efficiency heavy-flavor tags.
\item {\bf Upsilon spectroscopy:} Spectroscopy of the $\Upsilon$ family allows
to study the effects of color-screening as a function of size and 
binding energy of the quarkonium probes and as a function of density and size
of the medium for collisions of varying centrality. Separation of the three 
$\Upsilon$ states requires very good momentum resolution in the intermediate
\pt range and electron/hadron separation, thereby testing the performance of 
the outer tracker and electromagnetic calorimetry. 
\end{enumerate}

Within the collaboration, we established three Topical Groups to work 
on the science, analysis and performance evaluation related to each of the 
three science drivers. The Topical Groups worked with the software and simulation
team on one side and the subdetector project teams on the other side to 
establish the science impact of the individual re-scoping options.





