\section*{Thinned Outer HCal}
\label{ohcal_thin}

\textbf{Cost delta: -\$0.4M}

We investigated reducing the thickness of the outer HCal (nominal
depth of 82.5~cm) by 20~cm, or approxiately one nuclear interaction
length.  Several possible effects were investigated through full
\textsc{GEANT4} simulations of the calorimeter response.  It should be
noted that the calorimeter simulations have recently been compared to
data from the FNAL test beam and ahve been found to reproduce the
measured response quite well (put a number in here).  We looked at the
effect of thinning on the jet energy response, high-$z$ fragmentation
functions, and triggering in $pp$ and $pAu$.

From a physics perspective, the effect of thinning the outer HCal by
this amount seems to be moderate, and considered by itself, a thinner
outer HCal would still enable the core elements of the sphenix physics
program to be carried out.

From the perspective of the project, the effect of this modification
would be extremely significant.  The OHCal is the structural backbone
of the sPHENIX structure.  Reducing the radial thickness will
significantly reduce the ability of the OHCal to support the detector.
The finite element analysis would need to be redone. The tilt angle
reanalyzed and likely changed. The prototyping and test beam would
have to be redone.  It would make no sense to build the full scale
mechanical prototype now.  The IHCal will still need some redesign due
to the changes in the EMCal. We’ll want the $\phi$ segmentation to
match between the two for mechanical reasons.

Sets the OHCal engineering back as much as twelve months and requires
R\&D be redone.

\subsection*{Considerations}

In the nominal design, the $5.5 \lambda$ total depth of the
calorimeter stack (at $\eta = 0$) is distributed as $1 \lambda$
(EMCal), $1 \lambda$ (inner HCal), and $3.5 \lambda$.  If the outer
HCal is reduced to $2.5 \lambda$, one has to consider the effects if
one or more of the other calorimeters in the stack is also modified.
If the inner HCal is not installed, the depth of the calorimeter stack
is reduced by an additional $1 \lambda$ over the whole acceptance,
falling to $2.5 \lambda$ at midrapidity.

If the EMCal acceptance is cut to $|\eta| < 0.7$, the thickness of the
calorimeter stack in the range $0.7 < \eta < 1.1$ becomes thinner by
$1 \lambda$, but the effect in that range (aside from the lack of
EMCal coverage) is moderated by the increasing thickness of the
calorimeters along lines of constant $\eta$ due to their rectangular
profile.


