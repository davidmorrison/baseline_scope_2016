\section{DAQ/Trigger}
\label{daq}

\textbf{Cost delta: -\$1.5M}

The costs associated with the DAQ and Trigger are an area where
significant reductions are achievable.  Although we think the
\$0.5M trigger detector is a suitable target for a non-DOE, possibly
non-US, contribution, for the purposes of answering this charge, we do
not assume that that will happen.  Each of the RHIC experiments has
built trigger detectors that would suit the needs of sPHENIX, and some
of these detectors, such as the trigger detector used by PHOBOS, are
currently unused and available.  We would repurpose one of these
detectors for use in sPHENIX and reduce the M\&S costs accordingly.

We would reuse the infrastructure currently in place in the PHENIX
counting house, consisting of data collection modules (DCMIIs),
subevent buffers (SEBs), assembly and trigger processors (ATPs), a
high throughput networking switch and several racks of computers.
Some amount of new equipment would be needed, such as new level-1 and
global trigger boards.  Additional savings could be achieved by
deferring the event building to the offline stage. In this scenario,
the data would be written to disk at the stage of the SEBs. The event
fragments would not get assembled into full events until later, except
for a small fraction of them for online monitoring purposes.

This strategy does mean, absent identifying new funds to renew the
computing infrastructure, starting a new experiment with computing
that will be by then several years old and well beyond its operational
design lifetime and incurring an increased risk to experimental
operations as a result.



